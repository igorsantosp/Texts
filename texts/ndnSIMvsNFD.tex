\documentclass[preprint,12pt]{elsarticle}
%% The graphicx package provides the includegraphics command.
\usepackage{graphicx}
%% The amssymb package provides various useful mathematical symbols
\usepackage{amssymb}
%% The amsthm package provides extended theorem environments
%% \usepackage{amsthm}

%% The lineno packages adds line numbers. Start line numbering with
%% \begin{linenumbers}, end it with \end{linenumbers}. Or switch it on
%% for the whole article with \linenumbers after \end{frontmatter}.
\usepackage{lineno}
\journal{Journal Name}

\begin{document}
\begin{frontmatter}

%% Title, authors and addresses
\title{Tool for NDN architectures: ndnSIM vs. NFD}
\author{Igor}
\date{February 2017}
\address{Belém, Brazil}
\end{frontmatter}
\section{Introduction}
\label{S:1}
ndnSIM and NFD are so far the most relevant and used tools in the research and development of NDN networks, each one have its own  applications, even usually used together for realystic simulations, they can be used sepparately having each one a different function
\section{Main Conceipts}
\label{S:2}
\subsection{ndnSIM}
ndnSIM is a network simulator based on the ns-3 simulator, one of the most porwerful tool in simulating networks. ndnSIM can be described as a framework build over ns-3 to simulate NDN networks, ndnSIM have its applications written in c++ and even though its a application built to simulate ndn networks it uses NFD as it core to implement NDN function such as PIT's, FIB's and Content Store function
\end{document}